\chapter{Introduction} \label{cha:introduction}

In the development of new interfaces for the vehicles, or simply in the study of 
the influence in driving performance of some factor of interest, an evaluation
has to be performed. There are different ways to measure the driving quality of
a person under the interest conditions. In particular, there are two important
measurements for this task, the Time to Headway (THW) and the standard deviation
of lateral position (SDLP).

To calculate these measurements is necessary to use points of reference. In the
case of the Time to Headway the point of reference are the vehicles driving in
front of the testing car. In the case of the SDLP measurement, the lanes on the
road are used as the reference point to obtain the lateral position.

In a virtual environment is easy to obtain the exact position of the vehicles
and the lanes on the road, but in the real world is not an easy task. In real
world applications, sensors are used to estimate the position of these reference
points. Unfortunately, sensor measurements generally have some error associated
which is directly associated to the price of the sensor. Expensive sensors like
for example laser scanners are very accurate for distance calculation but at the
same time their price is very high. On the other hand, normal color cameras are
getting more popular to be used as sensors in real world reconstruction due to
its low cost and the recent advances in Computer Vision and Image Processing.

In particular, in the recent years a great number of applications using the
Smartphones cameras have been created. For example, augmented reality
applications that superpose a virtual environment by building references with
the real world through the camera.

Here is where the idea of using a similar application for the problem of
evaluating the driving quality shows up. In particular, to create an application
that uses the camera together with Computer Vision technique to calculate the
THW and SDLP measurements in real-time.

This project is precisely this idea taken to practice. The goal of the current
work is to develop an application
for the Android operative system used by Smartphones and mobile devices for
calculating these two measurements. The project was divided in two modules, one
for each measurement. The first one is the Vehicle Detection Module and is the
one responsible for calculate the THW. The second one is called Lane Detection
Module and is the responsible for the detection of the road lanes and the SDLP
calculation.
