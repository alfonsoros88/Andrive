\chapter{Implementation and experiments}  
\label{cha:android}

Nowadays the Android operative system is the most popular platform for
Smartphones and mobile devices. Due to this fact, there is a great interest in
the development of all sort of applications and with it, many tools have
been adapted to this platform for the use of the developers. One of these tools
is the OpenCV library for computer vision by \cite{opencv}.

This powerful library have already implemented the most common computer vision
algorithms and methods. In particular we can find the Machine Learning approach
for object detection in images explained in the previous
section~\ref{sec:objectdetec}.  It was perfectly suited by the implementation of this
project.

In this chapter is going to be presented the details about the Android
Application and the qualitative results on the performance of the application.

\section{The Android Application} % (fold)
\label{sec:androidapp}

In this section, the structure of the application is going to be presented. The
functionality of the main components is going to be explained aiming for future
improvements. In the figure~\ref{fig:andrivedir} is illustrated the directory
structure of the application.

\begin{figure}
    \centering\small
    \begin{minipage}[t]{0.6\linewidth}
        \dirtree{%
            .1 Andrive/. 
            .2 jni/. 
            .3 Android.mk. 
            .3 tum\_andrive\_DetectionBasedTracker.h.
            .3 detector\_native\_source.cpp.
            .2 src/. 
            .3 tum/.
            .4 andrive/.
            .5 Andrive.java.
            .5 GPSListener.java.
            .5 DetectionBasedTracker.java.
            .2 res/. 
            .3 raw/. 
            .4 cascade.xml. 
        }
    \end{minipage}
\caption{Directory structure of the Android application}
\label{fig:andrivedir}
\end{figure}

Maybe the most important file is the \textit{cascade.xml}. This file contains
the learned classifier used for image classification and vehicle detection. This
basically implies that it can be easily substituted by any other classifier for
either improving the classification or detect other objects.

Secondly, we have the source files under the \textit{src} directory. For the
vehicle detection module we have basically two files. The
\textit{DetectionBasedTracker.java} which defines the class with the same name.
This class implements the image classification task. It reads the
\textit{cascade.xml} file and use it to classified the images in the way
described in the chapter~\ref{kap:vehicle-detection}. The second file is the
\textit{GPSListener.java} which implements the class in charge of processing the
GPS signal and extract the speed for the calculation of the Time Headway
measurement (THM). 

Finally in the \textit{jni} directory are the native source files used by the
Java Native Interface (JNI). This native source is intended to accelerate the
task of image classification by compiling the methods in the C++ programming
language for the specific architecture. The class \textit{DetectionBasedTracker} 
can call these methods and also their equivalent in the Java programming
language.

% section androidapp (end)

\section{Training the classifier} % (fold)
\label{sec:trainingClasssifier}

In these section is going to be presented the details about the training phase.
The first step is to form the training set. We used a training set combining the
work of Markus Weber and the TME dataset by \cite{tme}. We have gather about 430
positive images and 3000 negative images.

Due to the Adaboost learning algorithm, a proportion of two negative samples for
each positive is recommendable.

% section trainingClasssifier (end)

\section{Performance and running time} % (fold)
\label{sec:running-time}

In this section is going to be discuss all the factors that directly affect the
running time and the performance. 

% section Performance and running time (end)


